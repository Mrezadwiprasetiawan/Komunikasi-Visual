\documentclass[a4paper,12pt]{article}
\usepackage[utf8]{inputenc}
\usepackage{graphicx}
\usepackage{geometry}
\geometry{left=2.5cm, right=2.5cm, top=3cm, bottom=3cm}
\usepackage{amsmath}
\usepackage{amssymb}
\usepackage[indonesian]{babel}
\usepackage{xcolor} % untuk text color
\usepackage{hyperref} % Untuk hyperlink
\usepackage{tocloft} % Untuk mengatur daftar isi dari suatu daftar(bukan table of content

% Untuk mengubah format penomoran menjadi arab
\renewcommand\thesection{\arabic{section}}
% Untuk mengubah format penomoran agar mengabaikan label dari format penomoran section
\renewcommand\thesubsection{\arabic{section}.\arabic{subsection}}

% Untuk daftar pustaka pemformatan nomor dihapus
\newcommand{\nonumsection}[1]{\section*{#1}
\addcontentsline{toc}{section}{#1}
}

\usepackage{titling}

\title{
  \vspace{-3cm}
  \centering
  \vspace{1cm}
  \textbf{Teori warna}\\
  \large Mata Kuliah: \textbf{Komunikasi Visual}
}
\author{\textbf{M Reza Dwi Prasetiawan}}
\date{\today}

\begin{document}
\pagenumbering{gobble}
\begin{titlepage}
  \maketitle
  \vfill
  \begin{center}
    \includegraphics[width=\textwidth]{resources/logo.png}\\  % Center the logo and keep aspect ratio
    \large
    Teknologi Informasi\\
    Institut Teknologi dan Sains\\
    Nahdatul Ulama Lampung\\
    Dosen Pembimbing: \textbf{Dewi Puspitasari, M.Sos}
  \end{center}
\end{titlepage}

\newpage
\pagenumbering{arabic}
% Insert table of contents
\tableofcontents
\newpage
\section{Psikologi Warna}
\section{Skema Warna}
\section{Aplikasi Warna Dalam Desain}
\nonumsection{Daftar Pustaka}
\begin{enumerate}
  \item \textcolor{blue}{\href{%link}{%visibleText}}
\end{enumerate}
\end{document}